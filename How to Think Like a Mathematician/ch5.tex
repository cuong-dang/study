\documentclass{article}
\usepackage[margin=1in]{geometry}
\usepackage{amsmath, amsthm, enumitem}

\begin{document}

\section*{Chapter 5}
\subsection*{Exercises 5.3}

\begin{enumerate}[label=(\roman*)]
    \item \begin{proof}
              For all $a, b > 0$, we have
              \begin{align*}
                  (\sqrt{a} - \sqrt{b})^2 & \ge 0                                   \\
                  a - 2\sqrt{ab} + b      & \ge 0
                                          & \text{(expand)}                         \\
                  a + b                   & \ge 2\sqrt{ab}
                                          & \text{(add $2\sqrt{ab}$ to both sides)} \\
                  \frac{a+b}{2}           & \ge \sqrt{ab}
                                          & \text{(divide both sides by 2).}
              \end{align*}
          \end{proof}
    \item \begin{proof}
              For all positive integers $a$, $b$, and $c$,
              \begin{align*}
                  (a-b)^2 + (b-c)^2 + (c-a)^2          & \ge 0                            \\
                  2(a^2 + b^2 + c^2) - 2(ab + bc + ca) & \ge 0
                                                       & \text{(expand and group terms)}  \\
                  2(a^2 + b^2 + c^2)                   & \ge 2(ab + bc + ca)
                                                       & \text{(move terms)}              \\
                  a^2 + b^2 + c^2                      & \ge ab + bc + ca
                                                       & \text{(divide both sides by 2).} \\
              \end{align*}
          \end{proof}
    \item $f$ is a 3-cycle function, e.g., $f^1(x) = f^4(x) = f^7(x) = \ldots$. In general,
          $f^r = f^n$ if $r \mod 3 = n \mod 3$ for all integers $r, n, 1 \le n \le 3, r > 3$.
          Therefore, $\displaystyle f^{653}(x) = f^2(x) = -\frac{1-x}{x}$ and
          $\displaystyle f^{653}(56) = \frac{55}{56}$.
    \item \begin{enumerate}[label=(\alph*)]
              \item We want to compare $\sqrt[7]{7!}$ with $\sqrt[8]{8!}$. Examining the difference
                    between the natural logarithms of both numbers,
                    \begin{align*}
                          & \ln{\sqrt[8]{8!}} - \ln{\sqrt[7]{7!}}                         \\
                        = & \frac{1}{8}\ln{8!} - \frac{1}{7}\ln{7!}                       \\
                        = & (\frac{1}{8}\ln{7!} + \frac{1}{8}\ln{8}) - \frac{1}{7}\ln{7!} \\
                        = & \frac{1}{8}\ln{8} - \frac{1}{56}\ln{7!}                       \\
                        = & \frac{7}{56}\ln{8} - \frac{1}{56}\ln{7!}                      \\
                        = & \frac{1}{56}\ln{8^7} - \frac{1}{56}\ln{7!}                    \\
                        = & \frac{1}{56}(\ln{8^7} - \ln{7!}).                             \\
                    \end{align*}
                    Since $\ln{8^7} > \ln{7!}$,
                    $\displaystyle \frac{1}{56}(\ln{8^7} - \ln{7!}) > 0$. Consequently,
                    $\ln{\sqrt[8]{8!}} - \ln{\sqrt[7]{7!}} > 0$. And thus,
                    $\sqrt[8]{8!} > \sqrt[7]{7!}$.
              \item \begin{proof}
                        \begin{align*}
                            2\times\sqrt{100000}          & < \sqrt{100001} + \sqrt{100000}                                                                     \\
                            1                             & < \frac{\sqrt{100001} + \sqrt{100000}}{2\times\sqrt{100000}}                                        \\
                            \sqrt{100001} - \sqrt{100000} & < \frac{(\sqrt{100001} + \sqrt{100000})\times(\sqrt{100001} - \sqrt{100000})}{2\times\sqrt{100000}} \\
                            \sqrt{100001} - \sqrt{100000} & < \frac{100001 - 100000}{2\times\sqrt{100000}}                                                      \\
                            \sqrt{100001} - \sqrt{100000} & < \frac{1}{2\times\sqrt{100000}}.
                        \end{align*}
                    \end{proof}
          \end{enumerate}
    \item The reasoning is not correct because adding the $\$2$ to the $\$27$ and then comparing
              them to the total $\$30$ does not make sense. The following equations show the
              relationships of the amounts.
              \begin{align*}
                  \underbrace{10 + 10 + 10}_{\text{what the friends initially paid}} & =
                  \underbrace{25}_{meal cost} + \underbrace{3}_{change} + \underbrace{2}_{waiter's} \\
                  \underbrace{9 + 9 + 9}_{\text{what the friends finally paid}}      & =
                  \underbrace{25}_{meal cost} + \underbrace{2}_{waiter's}.
              \end{align*}
              Thus, it does not make sense to add the $\$2$ on the right hand side to the $\$27$ on
          the left hand side.
    \item Initially, bottle $A$ has 1 litre of milk, and bottle $B$ has 1 litre of coffee. Suppose
          the spoonful amount of coffee poured from $B$ to $A$ is $\displaystyle \frac{1}{x}$.
          Afterwards, if the amounts of coffee and milk are mixed together and $A$, and then poured
          back into $B$ until $B$ has 1 litre of liquid, then the amount poured back is
          $\displaystyle \frac{1}{x}$, which contains both coffee and milk. In this mixture, there
          is $\displaystyle \frac{1}{1+\frac{1}{x}}$ part milk and
          $\displaystyle \frac{\frac{1}{x}}{1+\frac{1}{x}}$ part coffee. As a result, the final
          amount of coffee in A is
          \[
              \frac{1}{x} - \frac{\frac{1}{x}}{1+\frac{1}{x}}\times\frac{1}{x}
              = \frac{1}{x+1}.
          \]
          The final amout of milk in B is
          \[
              \frac{1}{1+\frac{1}{x}}\times\frac{1}{x}
              = \frac{1}{x+1}.
          \]
          Thus, the coffee part in A is equal to the milk part in B.
\end{enumerate}

\end{document}
