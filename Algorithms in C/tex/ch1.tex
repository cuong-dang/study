\documentclass{article}
\usepackage[margin=1in]{geometry}
\usepackage{amsmath, amsthm, enumitem, listings}

\everymath{\displaystyle}
\newtheorem{property}{Property}
\numberwithin{property}{section}

\title{Algorithms in C}
\author{}
\date{}

\begin{document}
\maketitle
\section{Introduction}
\subsection{Algorithms}
\subsection{A Sample Problem: Connectivity}
\subsubsection*{Exercises}
\begin{enumerate}[label=1-\arabic*]
      \item
            \begin{lstlisting}
          0-2
          1-4
          2-5
          3-6
          0-4
          6-0
          \end{lstlisting}
      \item
      \item Initialize \textit{count} to be N for N objects. Iterate through the pairs. For each
            \textit{union} call, decrease \textit{count} by 1.
\end{enumerate}

\subsection{Union-Find Algorithms}
\begin{property}
      The quick-find algorithm executes at least $MN$ instructions to solve a connectivity problem
      with $N$ objects that involves $M$ \textup{union} operations.
\end{property}
\begin{proof}
      For each of the $M$ \textit{union} operations, we iterate the \texttt{for} loop $N$ times. Each
      iteration requires at least one instruction (if only to check whether the loop is finished).
\end{proof}

\end{document}
