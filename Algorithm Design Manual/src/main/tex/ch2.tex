\documentclass{article}
\usepackage[margin=1in]{geometry}
\usepackage{amsmath, amsthm, enumitem, multicol}
\everymath{\displaystyle}

\begin{document}

\section*{Chapter 2}

\begin{enumerate}[label=2-\arabic*.]
      \item
      \item The loop can be expressed as
            \begin{align*}
                  \sum_{i=1}^{n}\sum_{j=1}^{i}\sum_{k=j}^{i+j}1
                   & = \sum_{i=1}^{n}\sum_{j=1}^{i}(i+1)
                   & \text{(innermost sum from $i$ to $2i$ has $i+1$ terms)} \\
                   & = \sum_{i=1}^{n}(i(i+1))                                \\
                   & = \sum_{i=1}^{n}i^2 + \sum_{i=1}^{n}i
                   & \text{(expand)}                                         \\
                   & = \frac{n(n+1)(2n+1)}{6} + \frac{n(n+1)}{2}
                   & \text{(apply explicit formulas)}                        \\
                   & = \frac{n(n+1)(n+2)}{3}
                   & \text{(simlify)}.                                       \\
            \end{align*}
            Thus, the expression is $\mathcal{O}(n^3)$.
      \item The loop can be expressed as
            \begin{align*}
                  \sum_{i=1}^{n}\sum_{j=1}^{i}\sum_{k=j}^{i+j}\sum_{l=1}^{i+j-k}1
                   & = \sum_{i=1}^{n}\sum_{j=1}^{i}\sum_{k=j}^{i+j}(i+j-k)              \\
                   & = \sum_{i=1}^{n}\sum_{j=1}^{i}\sum_{k=1}^{i}k
                   & \text{($\sum_{k=start}^{end}(end-k) = \sum_{k=1}^{end-start}(k)$)} \\
                   & = \sum_{i=1}^{n}\sum_{j=1}^{i}\frac{i(i+1)}{2}                     \\
                   & = \sum_{i=1}^{n}\frac{i^2(i+1)}{2}                                 \\
                   & = \frac{1}{2}(\sum_{i=1}^{n}i^3 + \sum_{i=1}^{n}i^2)               \\
                   & = \frac{1}{2}(\frac{n^2(n+1)^2}{4} + \frac{n(n+1)(2n+1)}{6}.
            \end{align*}
            Thus, the expression is $\mathcal{O}(n^4)$.
      \item The loop can be expressed as
            \begin{align*}
                  \sum_{i=1}^{n}\sum_{j=i+1}^{n}\sum_{k=i+j-1}^{n}1
                   & = \sum_{i=1}^{n}\sum_{j=i+1}^{n}(i+j-1)                \\
                   & = \sum_{i=1}^{n}\left(\sum_{j=i+1}^{n}i +
                  \sum_{j=i+1}^{n}j -
                  \sum_{j=i+1}^{n}1\right)                                  \\
                   & = \sum_{i=1}^{n}\left((n-i)i +
                  \left((n-i)i + \frac{(n-i)(n-i+1)}{2}\right) -
                  (n-i)\right)                                              \\
                   & = \sum_{i=1}^{n}\left(\frac{1}{2}(n-i)(5i-n+1)\right).
            \end{align*}
            Thus, the expression is $\mathcal{O}(n^3)$.
      \item
      \item \begin{enumerate}[label=(\alph*)]
                  \item $2n$ multiplications. $n$ additions.
                  \item $2n$ multiplications.
                  \item Horner's method reduces the number of multiplications to $n$.
            \end{enumerate}
      \item
      \item \begin{enumerate}[label=(\alph*)]
                  \item Yes. Because $2^{n+1} = 2 \times 2^n \le c2^n$  when $c \ge 2$ and $n > 0$.
                  \item No. Because $\forall c > 0, 2^n < 2^{2n}$ when $n > \log_2{c}$.
            \end{enumerate}
      \item \begin{enumerate}[label=(\alph*)]
                  \item $\log{n^2}$ is $\Theta(g(n))$ because $\log{n^2} = 2\log{n}$.
                  \item $\sqrt{n}$ is $\Omega(\log{n^2})$ because
                        $\displaystyle \lim_{n \to \infty}\frac{\sqrt{n}}{\log{n^2}} = \infty$.
                  \item $\log^2{n}$ is $\Omega(\log{n})$ because when $c = 1$, $f(n) \ge g(n)$ when
                        $n \ge 2$.
                  \item $n$ is $\Omega(\log^2{n})$ because
                        $\displaystyle \lim_{n \to \infty}\frac{n}{\log^2{n}} = \infty$.
                  \item $n\log{n} + n$ is $\Omega(\log{n})$.
                  \item 10 is $\Theta(\log{10})$.
                  \item $2^n$ is $\Omega(10n^2)$.
                  \item $2^n \in \mathcal{O}(3^n)$ because $2^n \le 3^n$ $\forall n \ge 1$.
            \end{enumerate}
      \item \begin{enumerate}[label=(\alph*)]
                  \item $g(n) \in \mathcal{O}(f(n))$.
                  \item $f(n) \in \mathcal{O}(g(n))$.
                  \item $f(n) \in \mathcal{O}(g(n))$.
                  \item $g(n) \in \mathcal{O}(f(n))$.
                  \item $f(n) \in \mathcal{O}(g(n))$ and $g(n) \in \mathcal{O}(f(n))$.
                  \item $f(n) \in \mathcal{O}(g(n))$.
            \end{enumerate}
      \item \begin{enumerate}[label=(\alph*)]
                  \item $\Theta(g(n))$.
                  \item $\mathcal{O}(g(n))$.
                  \item $\Theta(g(n))$.
                  \item $\Theta(g(n))$.
                  \item $\mathcal{O}(g(n))$.
            \end{enumerate}
      \item \begin{proof}
                  \begin{align*}
                        \lim_{n \to \infty}\frac{n^3-3n^2-n+1}{n^3} = 1.
                  \end{align*}
            \end{proof}
      \item \begin{proof}
                  \begin{align*}
                        \lim_{n \to \infty}\frac{n^2}{2^n} = 0.
                  \end{align*}
            \end{proof}
      \item \begin{proof}
                  \begin{align*}
                        \lim_{n \to \infty}\frac{n^2}{n^2+1} = 1.
                  \end{align*}
            \end{proof}
      \item \begin{enumerate}[label=(\alph*)]
                  \item \begin{itemize}
                              \item $\displaystyle \frac{(2n)^2}{n^2} = 4$ times slower.
                              \item $\displaystyle \frac{(n+1)^2}{n^2} =
                                          1 + \frac{2}{n} + \frac{1}{n^2}$ times slower.
                        \end{itemize}
                  \item \begin{itemize}
                              \item $\displaystyle \frac{(2n)^3}{n^3} = 8$ times slower.
                              \item $\displaystyle \frac{(n+1)^3}{n^3} =
                                          1 + \frac{3}{n} + \frac{3}{n^2} + \frac{1}{n^3}$
                                    times slower.
                        \end{itemize}
                  \item \begin{itemize}
                              \item $\displaystyle \frac{100(2n)^2}{100n^2} = 4$ times slower.
                              \item $\displaystyle \frac{100(n+1)^2}{100n^2} =
                                          1 + \frac{2}{n} + \frac{1}{n^2}$ times slower.
                        \end{itemize}
                  \item \begin{itemize}
                              \item $\displaystyle \frac{(2n)\log(2n)}{n\log{n}} =
                                          2 + \frac{2}{\log{n}}$ times slower.
                              \item $\displaystyle \frac{(n+1)\log(n+1)}{n\log{n}}$ times slower.
                        \end{itemize}
                  \item \begin{itemize}
                              \item $\displaystyle \frac{2^{2n}}{2^n} = 2^n$ times slower.
                              \item $\displaystyle \frac{2^{n+1}}{2^n} = 2$ times slower.
                        \end{itemize}
            \end{enumerate}
      \item \begin{enumerate}[label=(\alph*)]
                  \item 6,000,000.
                  \item 33,019.
                  \item 600,000.
                  \item $9 \times 10^{11}$.
                  \item 45.
                  \item 5.
            \end{enumerate}
      \item \begin{enumerate}[label=(\alph*)]
                  \item $\displaystyle \frac{3}{2}$.
                  \item 2.
                  \item 2.
            \end{enumerate}
      \item Because $f_1(n) \in \mathcal{O}(g_1(n))$, $\exists c_1, n_1$ such that
            $f_1(n) \le c_1g_1(n)$ for all $n \ge n_1$. Similarly, because
            $f_2(n) \in \mathcal{O}(g_2(n))$, $\exists c_2, n_2$ such that $f_2(n) \le c_2g_2(n)$
            for all $n \ge n_2$. Choose $c_3 = \max\lbrace  c_1, c_2 \rbrace$, and
            $n_3 = \max\lbrace  n_1, n_2 \rbrace$. When $n \ge n_3$, We have
            \begin{align*}
                  f_1(n) + f_2(n) & \le c_1g_1(n) + c_2g_2(n) \\
                                  & \le c_3g_1(n) + c_3g_2(n) \\
                                  & \le c_3(g_1(n) + g_2(n)).
            \end{align*}
            Thus, $f_1(n) + f_2(n) \in \mathcal{O}(g_1(n) + g_2(n))$ by definition.
      \item
      \item
      \item \begin{proof}
                  For all $n \ge 1$ and $i \le k$, $n^i \le n^k$. Because $a_i$ is any real number,
                  \begin{align*}
                        a_kn^k + \ldots + a_1n + a_0
                         & \le |a_k|n^k + \ldots + |a_1|n + |a_0|      \\
                         & \le |a_k|n^k + \ldots + |a_1|n^k + |a_0|n^k
                         & \text{($k$ is the highest degree)}          \\
                         & \le (|a_k| + \ldots + |a_1| + |a_0|)n^k.    \\
                  \end{align*}
                  Thus, if we choose $c = |a_k| + \ldots + |a_1| + |a_0|$ and $n_0 = 1$, then
                  $a_kn^k + \ldots + a_1n + a_0 \le cn^k$ for all $n \ge n_0$.
            \end{proof}
      \item By exercise 2-21, $(n+a)^b \in \mathcal{O}(n^b)$. The following is a proof for
            $(n+a)^b \in \Omega(n^b)$.
            \begin{proof}
                  Choose $n_0 = \max\left\{1, 2|a|\right\}$. Then for all $n \ge n_0$,
                  $n+a \ge n - |a| \ge n - n/2 = n/2$. Because $b > 0$, the function
                  $x \mapsto x^b$ is increasing on $[0,\infty)$; as a result, we have
                  $\displaystyle (n-a)^b \ge \left(\frac{n}{2}\right)^b = 2^{-b}n^b$. Thus, if we
                  choose $c = 2^{-b}$, by definition, $(n+a)^b \in \Omega(n^b)$.
            \end{proof}
      \item $n! \gg e^n \gg 2^n = 2^{n-1} \gg n-n^3+7n^5 \gg n^3 \gg n^2 + \log{n} = n^2 \gg
                  n^{1+e} \gg n\log{n} \gg n \gg \sqrt{n} \gg (\log{n})^2 \gg \ln{n} = \log{n} \gg
                  \log\log{n}$.
      \item $\pi^n \gg \sqrt{2^{\sqrt{n}}} \gg 2^{\log^4{n}} \gg n^4\binom{n}{n-4} \gg
                  n^{5(\log{n})^2}\gg \binom{n}{5} \gg \binom{n}{n-4} \gg n^\pi$.
      \item
      \item
      \item
      \item \begin{multicols}{4}
                  \begin{enumerate}[label=(\alph*)]
                        \item True.
                        \item False.
                        \item True.
                        \item False.
                        \item True.
                        \item True.
                        \item False.
                  \end{enumerate}
            \end{multicols}
      \item
            \begin{enumerate}[label=(\alph*)]
                  \item $f(n) = \Omega(g(n))$.
                  \item $f(n) = \mathcal{O}(g(n))$.
                  \item $f(n) = \Omega(g(n))$.
            \end{enumerate}
      \item
      \item \begin{multicols}{4}
                  \begin{enumerate}[label=(\alph*)]
                        \item No.
                        \item Yes.
                        \item Yes.
                        \item Yes.
                  \end{enumerate}
            \end{multicols}
      \item
      \item $f_4(n), f_2(n), f_1(n), f_3(n)$.
      \item (c) is true.
      \item \begin{multicols}{4}
                  \begin{enumerate}[label=(\alph*)]
                        \item $4^n$.
                        \item $n\log{n}$.
                        \item $(\log{n})^{10}$.
                        \item $n^{100}$.
                  \end{enumerate}
            \end{multicols}
      \item
            \begin{enumerate}[label=(\alph*)]
                  \item $\mathcal{O}, o$.
                  \item $\mathcal{O}, o$.
                  \item None.
                  \item $\mathcal{O}, o$.
                  \item To simplify, we will compare $n^{\ln{n}}$ and $(\ln{n})^n$. Rewriting them
                        in exponential form, $n^{\ln{n}} = e^{\ln{n}\ln{n}}$ and $(\ln{n})^n =
                              e^{(\ln\ln{n})n}$. Thus, we need to compare $(\ln{n})^2$ and
                        $n\ln\ln{n}$. Let $x = \ln{n}$, then $n = e^x$. We have, $(\ln{n})^2 = x^2$
                        and $n\ln\ln{n} = e^x\ln{x}$. It's easy to see that $x^2$ grows much slower
                        than $e^x\ln{x}$. Therefore, $(\ln{n})^2$ grows slower than $n\ln\ln{n}$.
                        And ultimately, $n^{\ln{n}}$ is $o((\ln{n})^n)$.
                  \item $\mathcal{O}, o$.
            \end{enumerate}
      \item $S_i = 3^{i-1}$ for $i = 1 \ldots n$. Proof by induction.
      \item \begin{align*}
                  \sum_{i=1}^{n}(in - i(i-1))
                   & = n\sum_{i=1}^{n}i - \sum_{i=1}^{n}i^2 + \sum_{i=1}^{n}i        \\
                   & = n\frac{n(n+1)}{2} - \frac{n(n+1)(2n+1)}{6} + \frac{n(n+1)}{2} \\
                   & = \frac{n^3+3n^2+2n}{6}.
            \end{align*}
      \item Subtract the sum of the numbers in the array from $\frac{(n+1)(n+2)}{2}$.
      \item \begin{align*}
                  \sum_{i=1}^{n}\sum_{j=i}^{2i}1 & = \sum_{i=1}^{n}\sum_{j=1}^{i+1}1  \\
                                                 & = \sum_{i=1}^{n}(i+1)              \\
                                                 & = \sum_{i=1}^{n} + \sum_{i=1}^{n}1 \\
                                                 & = \frac{n(n+1)}{2} + n             \\
                                                 & = \frac{n(n+3)}{2}.
            \end{align*}
      \item \begin{align*}
                  \sum_{i=1}^{\frac{n}{2}}\sum_{j=i}^{n-i}\sum_{k=1}^{j}1
                   & = \sum_{i=1}^{\frac{n}{2}}\sum_{j=i}^{n-i}j                       \\
                   & = \sum_{i=1}^{\frac{n}{2}}\sum_{j=i}^{n-i}j                       \\
                   & = \sum_{i=1}^{\frac{n}{2}}(\sum_{j=1}^{n-i}j - \sum_{j=1}^{i-1}j) \\
                   & = \sum_{i=1}^{\frac{n}{2}}\frac{n(n-2i+1)}{2}                     \\
                   & = \frac{n}{2}\sum_{i=1}^{\frac{n}{2}}(n-2i+1)                     \\
                   & = \frac{n}{2}(\frac{n^2}{2} -
                  \frac{n}{2}\left(\frac{n}{2}+1\right) + \frac{n}{2})                 \\
                   & = \frac{n^3}{8}.
            \end{align*}
      \item $\mathcal{O}(nb^n)$.
      \item
      \item
      \item \begin{proof}
                  Let $n \ge 1$ and $k = \lfloor\log{n}\rfloor$. By the definition of the floor
                  function, $k \le \log{n} < k+1$; hence, $2^k \le n < 2^{k+1}$. Adding 1 gives
                  $2^k + 1 \le n+1 \le 2^{k+1}$. Because the binary logarithm $\log$ is strictly
                  increasing, taking $log$ yields $\log{2^k+1} \le \log(n+1) \le k+1$. Since
                  $2^k + 1 > 2^k$, $\log{2^k+1} > k$. As a result, $k < \log(n+1) \le k+1$. Because
                  $k+1$ is the only integer in the open interval, $\lceil\log(n+1)\rceil = k + 1 =
                        \lfloor\log{n}\rfloor + 1$.
            \end{proof}
      \item \begin{proof}
                  For $n \ge 1$, $n$ can be expressed as $n = 2^kb_k + 2^{k-1}b_{k-1} + \ldots +
                        2^0b_0$ where $b_k = 1$ and $b_i$ equals 0 or 1 for $0 \le i \le k-1$. Thus,
                  the number of bits necessary to represent $n$ is $k+1$. We now try to prove that
                  $k+1 = \lfloor\log{n}\rfloor+1$. Based on the above expression, we have
                  \begin{align*}
                        2^k & \le n         & \le 2^{k+1} - 1 \\
                        2^k & < n + 1       & \le 2^{k+1}     \\
                        k   & < \log(n + 1) & \le k+1.
                  \end{align*}
                  From the definition of the ceiling function, we have $k+1 =
                        \lceil\log(n+1)\rceil$. Hence, $k+1 = \lfloor\log{n}\rfloor + 1$, based on
                  the previous exercise.
            \end{proof}
      \item $\mathcal{O}(n\log\sqrt{n})$ is still $\Omega(n\log{n})$.
      \item If $n$ is known, the algorithm is as follows.
            \begin{enumerate}[label=\arabic*.]
                  \item Randomly select an element whose index is between 0 to $n-1$.
                  \item Add this element to the result, remove this element from the original list.
                  \item Randomly select an element whose index is between 0 to $n-2$.
                  \item Continue until we select $k$ elements.
            \end{enumerate}
            \begin{proof}
                  The probability of an element that is not selected is
                  \[\frac{n-1}{n}\times\frac{n-2}{n-1}\times\ldots\times\frac{n-k}{n-k+1} =
                        \frac{n-k}{n}\]
                  Thus, the probability of a selected element is $\frac{k}{n}$.
            \end{proof}
            If $n$ is not known, then, assuming we receive a stream of $n$ elements, one at a time,
            the algorithm is as follows.
            \begin{enumerate}[label=\arabic*.]
                  \item Select first $k$ elements.
                  \item For the $i^{th}$ element where $i > k$, randomly choose a number $r$ between
                        0 and $i-1$. If $r \le i-1$, replace the element at index $r$ with the
                        $i^{th}$ element.
            \end{enumerate}
            \begin{proof}
                  For a given $i^{th}$ element, the probability for its to be selected is
                  $\frac{k}{i}$. Let's say an $i_0^{th}$ element was selected. Its probability to
                  still be selected after $j$ more elements, where $i_0 + j = n$, is
                  \[\frac{k}{i_0}\times\frac{i_0}{i_0+1}\times\ldots\times\frac{i_0+j-1}{i_0+j} =
                        \frac{k}{i_0+j} = \frac{k}{n}.\]
            \end{proof}
      \item
      \item
      \item
      \item Weigh 3 vs. 3. If they are balanced, weigh the other two. If one group is lighter, weigh
            any two in that group.
      \item Assume that only pairwise merges are allowed, not k-way merges. A total of $n-1$ merges
            are required. At the first step, there are $\binom{n}{2}$ ways to merge. At the second
            step, there are $\binom{n-1}{2}$ ways to merge. Thus, the total number of ways to merge
            is
            \[\prod_{i=0}^{n-2}\binom{n-i}{2} = \frac{n!(n-1)!}{2^n}.\]
\end{enumerate}

\end{document}
