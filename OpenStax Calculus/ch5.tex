\documentclass{article}
\usepackage[margin=1in]{geometry}
\usepackage{amsmath, amsthm, enumitem}

\newtheorem*{theorem*}{Theorem}

\begin{document}

\section*{5.3 | The Fundamental Theorem of Calculus}
\begin{theorem*}[The Mean Value Theorem for Integrals]
    If $f(x)$ is continuous over an interval $\left[a, b\right]$, then there is at least one point $c
        \in \left[a, b\right]$ such that
    \[
        f(c) = \frac{1}{b-a}\int_{a}^{b}f(x)dx.
    \]
    This formula can also be stated as
    \[
        \int_{a}^{b}f(x)dx = f(c)(b-a).
    \]
\end{theorem*}
\begin{proof}
    Since $f(x)$ is continuous on $\left[a, b\right]$, by the extreme value theorem, it assumes
    minimum and maximum values---$m$ and $M$, respectively---on $\left[a, b\right]$. Then, for all
    $x$ in $\left[a, b\right]$, we have $m \le f(x) \le M$. Therefore, by the comparison theorem, we
    have
    \[
        m(b-a) \le \int_{a}^{b}f(x)dx \le M(b-a).
    \]
    Dividing by $b-a$ gives us
    \[
        m \le \frac{1}{b-a}\int_{a}^{b}f(x)dx \le M.
    \]
    Sine $\displaystyle \frac{1}{b-a}\int_{a}^{b}f(x)dx$ is a number between $m$ and $M$, and since
    $f(x)$ is continuous and assumes the values $m$ and $M$ over $\left[a, b\right]$, by the
    Intermediate Value Theorem, there is a number $c$ over $\left[a, b\right]$ such that
    \[
        f(c) = \frac{1}{b-a}\int_{a}^{b}f(x)dx,
    \]
    and the proof is complete.
\end{proof}
\end{document}
