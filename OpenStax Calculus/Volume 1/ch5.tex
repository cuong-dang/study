\documentclass{article}
\usepackage[margin=1in]{geometry}
\usepackage{amsmath, amsthm, enumitem}

\everymath{\displaystyle}
\newtheorem*{theorem*}{Theorem}
\newtheorem*{lemma*}{Lemma}

\begin{document}

\section*{5.3 | The Fundamental Theorem of Calculus}
\begin{theorem*}[The Mean Value Theorem for Integrals]
    If $f(x)$ is continuous over an interval $\left[a, b\right]$, then there is at least one point $c
        \in \left[a, b\right]$ such that
    \[
        f(c) = \frac{1}{b-a}\int_{a}^{b}f(x)dx.
    \]
    This formula can also be stated as
    \[
        \int_{a}^{b}f(x)dx = f(c)(b-a).
    \]
\end{theorem*}
\begin{proof}
    Since $f(x)$ is continuous on $\left[a, b\right]$, by the extreme value theorem, it assumes
    minimum and maximum values---$m$ and $M$, respectively---on $\left[a, b\right]$. Then, for all
    $x$ in $\left[a, b\right]$, we have $m \le f(x) \le M$. Therefore, by the comparison theorem, we
    have
    \[
        m(b-a) \le \int_{a}^{b}f(x)dx \le M(b-a).
    \]
    Dividing by $b-a$ gives us
    \[
        m \le \frac{1}{b-a}\int_{a}^{b}f(x)dx \le M.
    \]
    Sine $\frac{1}{b-a}\int_{a}^{b}f(x)dx$ is a number between $m$ and $M$, and since
    $f(x)$ is continuous and assumes the values $m$ and $M$ over $\left[a, b\right]$, by the
    Intermediate Value Theorem, there is a number $c$ over $\left[a, b\right]$ such that
    \[
        f(c) = \frac{1}{b-a}\int_{a}^{b}f(x)dx,
    \]
    and the proof is complete.
\end{proof}

\begin{theorem*}[Fundamental Theorem of Calculus, Part 1]
    If $f(x)$ is continuous over an interval $\left[a, b\right]$, and the function $F(x)$ is defined
    by
    \[F(x) = \int_{a}^{x}f(t)dt,\]
    then $F'(x) = f(x)$ over $\left[a, b\right]$.
\end{theorem*}
\begin{proof}
    Applying the definition of the derivative, we have
    \begin{align*}
        F'(x) & = \lim_{h \to 0}\frac{F(x+h) - F(x)}{h}                                           \\
              & = \lim_{h \to 0}\frac{1}{h}\left(\int_{a}^{x+h}f(t)dt - \int_{a}^{x}f(t)dt\right) \\
              & = \lim_{h \to 0}\frac{1}{h}\left(\int_{a}^{x+h}f(t)dt + \int_{x}^{a}f(t)dt\right) \\
              & = \lim_{h \to 0}\frac{1}{h}\int_{x}^{x+h}f(t)dt.
    \end{align*}
    $\frac{1}{h}\int_{x}^{x+h}f(t)dt$ is the average value of function $f(x)$ over the interval
    $\left[x, x+h\right]$. Therefore, by The Mean Value Theorem for Integrals, there is some number
    $c$ in $\left[x, x+h\right]$ such that
    \[\frac{1}{h}\int_{x}^{x+h}f(x)dx = f(c).\]
    In addition, since $c$ is between $x$ and $x+h$, $c$ approaches $x$ as $h$ approaches zero.
    Also, since $f(x)$ is continuous, we have $\lim_{h \to 0}f(c) = \lim_{c \to x}f(c) = f(x)$.
    Putting all these pieces together, we have
    \begin{align*}
        F'(x) & = \lim_{h \to 0}\frac{1}{h}\int_{x}^{x+h}f(x)dx \\
              & = \lim_{h \to 0}f(c)                            \\
              & = f(x),
    \end{align*}
    and the proof is complete.
\end{proof}
\begin{lemma*}
    Let $c$ be a value between the interval $\left[x, x+h\right]$, $c$ approaches $x$ as $h$
    approaches 0.
\end{lemma*}
\begin{proof}
    We can regard $c$ as defined by some function of $h$, $c = f(h)$. We are trying to prove
    \[\lim_{h \to 0}f(h) = x.\]
    By the definition of limit, we want to show that for all $\epsilon > 0$, there exists a $\delta
        > 0$ such that if $0 < |h| < \delta$, then $|f(h) - x| < \epsilon$. Because $c$ is a value
    between $\left[x, x+h\right]$, if $h > 0$,
    \begin{align*}
        x & \le  c        & \le & x+h  \\
        x & \le  f(h)     & \le & x+h  \\
        0 & \le  f(h)-x   & \le & h    \\
        0 & \le  |f(h)-x| & \le & |h|.
    \end{align*}
    Similarly, if $h < 0$,
    \begin{align*}
        x +h & \le  c        & \le & x    \\
        x+h  & \le  f(h)     & \le & x    \\
        h    & \le  f(h)-x   & \le & 0    \\
        |h|  & \ge  |f(h)-x| & \ge & 0    \\
        0    & \le  |f(h)-x| & \le & |h|.
    \end{align*}
    In either case, $|f(h) - x| \le |h|$. Choose $\delta = \epsilon$. Consequently, if
    $0 < |h| < \delta$, then $|f(h) - x| \le |h| < \delta = \epsilon.$ By definition,
    $\lim_{h \to 0}f(h) = x$, or $\lim_{h \to 0}c = x$.
\end{proof}

\begin{theorem*}[The Fundamental Theorem of Calculus, Part 2]
    If $f$ is continuous over the interval $\left[a, b\right]$ and $F(x)$ is any derivative of
    $f(x)$, then
    \[\int_{a}^{b}f(x)dx = F(b) - F(a).\]
\end{theorem*}
\begin{proof}
    Let $P = \left\{x_i\right\}, i = 0, 1, \ldots, n$ be a regular partition of $\left[a, b\right]$.
    Then, we can write
    \begin{align*}
        F(b) - F(a) & = F(x_n) - F(x_0)                                                          \\
                    & = F(x_n) - F(x_{n-1}) + F(x_{n-1}) - F(x_{n-2}) + \ldots + F(x_1) - F(x_0) \\
                    & = \sum_{i=1}^{n}(F(x_i) - F(x_{i-1})).
    \end{align*}
    Now, we know $F$ is an antiderivative of $f$ over $\left[a, b\right]$, so by the Mean Value
    Theorem for $i = 0, 1, \ldots, n$ we can find $c_i$ in $\left[x_{i-1}, x_i\right]$ such that
    \[F(x_{i}) - F(x_{i-1}) = F'(c_i)(x_i - x_{i-1}) = f(c_i)\Delta{x}.\]
    Then, substituing into the previous equation, we have
    \[F(b) - F(a) = \sum_{i=1}^{n}f(c_i)\Delta{x}.\]
    Taking the limit of both sides as $n \to \infty$, we obtain
    \begin{align*}
        F(b) - F(a) & = \lim_{n \to \infty}\sum_{i=1}^{n}f(c_i)\Delta{x} \\
                    & = \int_{a}^{b}f(x)dx.
    \end{align*}
\end{proof}

\end{document}
