\documentclass{article}
\usepackage[margin=1in]{geometry}
\usepackage{amsmath, enumitem}

\everymath{\displaystyle}

\title{Parallel Computer Architecture}
\author{}
\date{}

\begin{document}
\maketitle
\section{Introduction}
\begin{enumerate}[label=1.\arabic*]
      \setcounter{enumi}{2}
      \item Suppose that $T_1 = 1$ and the sequential fraction is $s$, the parallelizable fraction is
            $1-s$. Since the parallelizable part is perfectly parallelizable on $P$ processors,
            \[T_p = s + \frac{1-s}{p}.\]
            The speedup then is,
            \[S = \frac{T_1}{T_p} = \frac{1}{s+\frac{1-s}{p}}.\]
            The upper bound is,
            \[\lim_{p \to \infty}S = \lim_{p \to \infty}\frac{1}{s+\frac{1-s}{p}} = \frac{1}{s}.\]
            This is reasonable because suppose the serial part of a program accounts for $\frac{1}{3}$
            of the program, then even when the parallelizable can be perfectly parallelized on an
            infinite number of processors, the total runtime cannot be less than $\frac{1}{3}$, which
            translates to a speedup factor of 3.
            \setcounter{enumi}{5}
      \item Given a shared memory location accessible through the variable \texttt{counter}, each
            process will execute a function using instruction \texttt{fetch\&inc} whose return value
            is used as the process's id. The final value of \texttt{counter} is the number of
            processes.
            \setcounter{enumi}{7}
      \item If we divide the 1024 rows into 64 blocks of 16 consecutive rows each, only the
            boundary rows between processors need to communicate. There are 63 such rows; thus, the
            total data traffic is $63 \times 1024 \times 8 = 516,096$ bytes.
      \item $P_{\text{pipelined}} = \max\left\{\frac{m}{100\times10^6}, T\right\}$
            \setcounter{enumi}{12}
      \item $\approx3.5$ $\mu$s.
\end{enumerate}
\end{document}
